\documentclass[12pt]{article}
\usepackage[utf8]{inputenc}
\usepackage[italian]{babel}
\usepackage{graphicx}
\usepackage{geometry}
\usepackage{url}
\usepackage[hang,flushmargin]{footmisc}
\usepackage{fancyhdr}
\usepackage{lastpage}
\usepackage{footnote}
\usepackage{float}
\usepackage{textcomp}
\usepackage[colorinlistoftodos]{todonotes}
%\usepackage[hidelinks]{hyperref}
\PassOptionsToPackage{hyphens}{url}\usepackage{hyperref}
\hypersetup{colorlinks, linkcolor=black, urlcolor=blue}
\newenvironment{sistema}% 
{\left\lbrace\begin{array}{@{}l@{}}}% 
	{\end{array}\right.}
%Layout di pagina
\geometry{
	a4paper,
	total={160mm,225mm},
	left=25mm,
	top=25mm
}

%intestazione e piè di pagina
\pagestyle{fancy}
\fancyhf{}
\renewcommand{\headrulewidth}{0.4pt}
\renewcommand{\footrulewidth}{0.4pt}
\setlength\headheight{57pt}
\lhead{Relazione del progetto \textbf{DevSpace}}
\lfoot{DevSpace}
\rfoot{Pagina \thepage\ di \pageref{LastPage}}


%\newcommand{\code}[1]{\texttt{#1}}
\interfootnotelinepenalty=10000
\DeclareTextFontCommand{\code}{\ttfamily\hyphenchar\font=45\relax}

\begin{document}

	\begin{titlepage} % Suppresses displaying the page number on the title page
	\begin{center} % Centre everything on the page

		%------------------------------------------------
		%	Headings
		%------------------------------------------------
		
		\textsc{\LARGE Università degli studi di Padova}\\[1.5cm]
		
		\textsc{\Large Corso di Laurea in Informatica}\\[0.5cm]
		
		\textsc{\Large Progetto di Ricerca Operativa}\\[0.5cm]
		

		
		\textsc{\large a.a. 2018/2019}\\[0.5cm] % Minor heading such as course title
		
		%------------------------------------------------
		%	Title
		%------------------------------------------------
		
		{\huge\ Relazione}\\[0.4cm] % Title of your document
		
		%------------------------------------------------
		%	Author(s)
		%------------------------------------------------
		
	 	\large
			\begin{tabular}{l r}
				Giacomo \textsc{Barzon} & 1143164 \\
			\end{tabular}
	\end{center}
\end{titlepage}
	%----------------------------------------------------------------------------------------

	\newpage
	
	\tableofcontents
	
	\newpage
	
	\section{Problema}
	Un azienda siderurgica produce tre delle principali componenti fondamentali per la realizzazione di motori per auto, le quali verranno chiamate come componenti A, B e C per comodità.\\
	Ogni componente per essere realizzata deve seguire uno specifico iter composto da 3 lavorazioni le quali devono essere eseguite obbligatoriamente in una specifica sequenza.\\
	All'interno della seguente tabella è possibile vedere, per ogni tipologia di componente e lavorazione, il numero di unità che è possibile completare in un ora.
	
	\begin{table}[H]
		\setlength{\tabcolsep}{15pt} % Default value: 6pt
		\renewcommand{\arraystretch}{2} % Default value: 1
		\begin{center}
			\begin{tabular}{|c|c|c|c|}
				\hline
				& \textbf{Lavorazione 1} & \textbf{Lavorazione 2} & \textbf{Lavorazione 3} \\ \hline
				\textbf{A} &      10      &      8      &    11        \\ \hline
				\textbf{B} &       8     &       7     &      9      \\ \hline
				\textbf{C} &        13    &      10      &    15       \\ \hline
			\end{tabular}
		\end{center}
	\end{table}

	Attualmente l'azienda possiede solamente una macchina per ogni tipologia di lavorazione. Ogni macchina può lavorare solo ed esclusivamente su una tipologia di componente per volta e per un massimo di 8 ore complessive a giorno.\\
	Ogni qualvolta sia necessario passare dalla lavorazione di un componente all'altro durante la giornata è necessario effettuare una configurazione del macchinario di circa un ora.\\
	Ogni componente non può subire due lavorazioni nello stesso giorno in quanto deve effettuare un periodo di riposo per evitare di alterarne la qualità.\\
	Tra una lavorazione e l'altra ogni componente deve effettuare un periodo di riposo, per questo motivo non è possibile effettuare effettuare due lavorazioni sulla stessa unità lo stesso giorno.
	Ogni macchina ha un costo orario, in particolare:
	\begin{itemize}
		\item il macchinario 1 costa 4 all'ora;
		\item il macchinario 2 costa 5 all'ora;
		\item il macchinario 3 costa 3 all'ora.
	\end{itemize}
	Attualmente l'azienda ha stipulato un contratto con un azienda la quale paga:
	\begin{itemize}
		\item 30 per ogni unità di componente 1
		\item 25 per ogni unità di componente 2
		\item 40 per ogni unità di componetne 3
	\end{itemize}
	Essa inoltre richiede una quantita minima di:
	\begin{itemize}
		\item 10 unità di componente 1;
		\item 15 unità di componente 2;
		\item 13 unità di componente 3.
	\end{itemize}
	Inoltre per ogni unità non recapitata l'azienda è costretta a pagare una penale di 35 euro.
	\pagebreak
	\section{Modello Matematico}
	\subsection{Variabili decisionali}
	\begin{itemize}
		\item \( x_{pmg} : p \in \{A, B, C\},\ m \in \{1, 2, 3\},\ g \in \{lun, mar, mer, gio, ven, sab, dom\}\): Numero di ore in cui vengono lavorati pezzi di tipo p all'interno della macchina m nel giorno g.
		\item \( y_{psg} : p \in \{A, B, C\},\ s \in \{1, 2, 3\},\ g \in \{lun, mar, mer, gio, ven, sab, dom\}\): Numero di pezzi di tipo p che hanno terminato la lavorazione di tipo s durante il giorno g.
		\item \( y_{pmg} : p \in \{A, B, C\},\ m \in \{1, 2, 3\},\ g \in \{lun, mar, mer, gio, ven, sab, dom\}\): Variabile binaria che vale:
		\[ 
			y_{pmg}=
			\begin{sistema} 
				\: 1 \quad se\ la\ macchina\ m\ ha\ lavorato\ a\ pezzi\ di\ tipo\ p\ il\ giorno\ g\\
				\: 0 \quad altrimenti \\ 
			\end{sistema} 
		\]
	\end{itemize}
	\subsection{Parametri}
	\begin{itemize}
		\item \( v_p : p \in \{A, B, C\} \): prezzo di vendita della componente p.
		\item \( c_m : m \in \{1, 2, 3\} \): costo orario della macchina m
		\item \( po_{pm} : p \in \{A, B, C\},\ m \in \{1, 2, 3\}\): produzione oraria della componente p all'interno della macchina m.
		\item \( m_p : p \in \{A, B, C\}\): numero minimo di componenti p richieste.
	\end{itemize}
\end{document}
