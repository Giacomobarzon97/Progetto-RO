\documentclass[12pt]{article}
\usepackage[utf8]{inputenc}
\usepackage[italian]{babel}
\usepackage{graphicx}
\usepackage{geometry}
\usepackage{url}
\usepackage[hang,flushmargin]{footmisc}
\usepackage{fancyhdr}
\usepackage{lastpage}
\usepackage{footnote}
\usepackage{float}
\usepackage{textcomp}
\usepackage[colorinlistoftodos]{todonotes}
%\usepackage[hidelinks]{hyperref}
\PassOptionsToPackage{hyphens}{url}\usepackage{hyperref}
\hypersetup{colorlinks, linkcolor=black, urlcolor=blue}

%Layout di pagina
\geometry{
	a4paper,
	total={160mm,225mm},
	left=25mm,
	top=25mm
}

%intestazione e piè di pagina
\pagestyle{fancy}
\fancyhf{}
\renewcommand{\headrulewidth}{0.4pt}
\renewcommand{\footrulewidth}{0.4pt}
\setlength\headheight{57pt}
\lhead{Relazione del progetto \textbf{DevSpace}}
\lfoot{DevSpace}
\rfoot{Pagina \thepage\ di \pageref{LastPage}}


%\newcommand{\code}[1]{\texttt{#1}}
\interfootnotelinepenalty=10000
\DeclareTextFontCommand{\code}{\ttfamily\hyphenchar\font=45\relax}

\begin{document}

	\begin{titlepage} % Suppresses displaying the page number on the title page
	\begin{center} % Centre everything on the page

		%------------------------------------------------
		%	Headings
		%------------------------------------------------
		
		\textsc{\LARGE Università degli studi di Padova}\\[1.5cm]
		
		\textsc{\Large Corso di Laurea in Informatica}\\[0.5cm]
		
		\textsc{\Large Progetto di Ricerca Operativa}\\[0.5cm]
		

		
		\textsc{\large a.a. 2018/2019}\\[0.5cm] % Minor heading such as course title
		
		%------------------------------------------------
		%	Title
		%------------------------------------------------
		
		{\huge\ Relazione}\\[0.4cm] % Title of your document
		
		%------------------------------------------------
		%	Author(s)
		%------------------------------------------------
		
	 	\large
			\begin{tabular}{l r}
				Giacomo \textsc{Barzon} & 1143164 \\
			\end{tabular}
	\end{center}
\end{titlepage}
	%----------------------------------------------------------------------------------------

	\newpage
	
	\tableofcontents
	
	\newpage
	
	\section{Problema}
	Si vuole massimizzare il numero di ore che uno studente universitario può investire nelle sue attività quotidiane permettendogli in questo modo di sprecarne il minimo possibile. Per fare ciò è necessario definire una routine settimanale che sfrutti il suo tempo al meglio.
	Le giornate dello studente sono suddivisibili principalmente in tre fasce orarie.
	\begin{itemize}
		\item {\textbf{Mattina}: va dalle 8:00 fino alle 13:00.}
		\item{\textbf{Pomeriggio}: va dalle 14:00 fino alle 19:00.}
		\item{\textbf{Sera}: va dalle 20:00 fino alle 23:00.}		
	\end{itemize}
	Di seguito è presente una lista delle principali attività che lo studente può compiere durante le sue giornate:
	\begin{itemize}
		\item \textbf{Seguire le lezioni}: Lo studente per poter studiare efficacemente deve seguire tutte le lezioni. Di seguito è presente una tabella in cui vengono indicate le fasce orarie in cui lo studente ha lezione.
		\item \textbf{Studiare}: Lo studente per superare con successo i vari esami universitari deve dedicare un minimo di 15 ore settimanali di studio. Lo studente inoltre per mantenere la concentrazione non può fare più di 2:00 di studio consecutive, e fare un minimo di 15 minuti di pausa tra una sessione di studio e l'altra. Tuttavia non può effettuare sessioni di studio di durata minore di 1:00 in quanto sarebbero di efficacia praticamente nulla.
		\item \textbf{Palestra}: Lo studente deve andare in palestra un minimo di tre volte a settimana. Per avere tempi di recupero tra un allenamento e l'altro lo studente non può andare in palestra se ci è già andato il giorno precedente. La palestra è chiusa la sera e nei weekend anche il pomeriggio. Per effettuare un allenamento lo studente impiega mediamente circa 2:00 ore.

	\end{itemize}
	Ogni volta che lo studente interrompe l'esecuzione di un attività per cominciarne un altra  è necessario spendere 15 minuti per la preparazione.

\end{document}
