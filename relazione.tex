\documentclass[12pt]{article}
\usepackage[utf8]{inputenc}
\usepackage[italian]{babel}
\usepackage{graphicx}
\usepackage{geometry}
\usepackage{url}
\usepackage{amsmath, amsfonts, amssymb}
\usepackage[hang,flushmargin]{footmisc}
\usepackage{fancyhdr}
\usepackage{lastpage}
\usepackage{footnote}
\usepackage{float}
\usepackage{textcomp}
\usepackage[colorinlistoftodos]{todonotes}
%\usepackage[hidelinks]{hyperref}
\PassOptionsToPackage{hyphens}{url}\usepackage{hyperref}
\usepackage{listings}
\usepackage{courier}
\usepackage{xcolor}
\definecolor{darkGreen}{rgb}{0.01, 0.75, 0.24}



\lstdefinelanguage{AMPL}{keywords={set,param,var,arc,integer,minimize,maximize,subject,to,node,sum,in,Current,complements,integer,solve_result_num,IN,contains,less,suffix,INOUT,default,logical,sum,Infinity,dimen,max,symbolic
		,Initial,div,min,table,LOCAL,else,option,then,OUT,environ,setof ,union,all,exists,shell_exitcodeuntil,binary,forall,solve_exitcodewhile ,by,if,solve_messagewithin,check,in,solve_result, prev, ord, first, member, and
	},sensitive=true,comment=[l]{\#}}

\lstset{frame=tb,
	language=AMPL,
	aboveskip=3mm,
	belowskip=3mm,
	showstringspaces=false,
	columns=flexible,
	basicstyle={\ttfamily},
    numbers=left,
	stepnumber=1,
	keywordstyle=\bfseries\color{blue},
	commentstyle=\color{darkGreen},
	stringstyle=\color{red},
	breaklines=true,
	breakatwhitespace=true,
	tabsize=3
}
\hypersetup{colorlinks, linkcolor=black, urlcolor=blue}
\newenvironment{sistema}% 
{\left\lbrace\begin{array}{@{}l@{}}}% 
	{\end{array}\right.}
%Layout di pagina
\geometry{
	a4paper,
	total={160mm,225mm},
	left=25mm,
	top=25mm
}

%intestazione e piè di pagina
\pagestyle{fancy}
\fancyhf{}
\renewcommand{\headrulewidth}{0.4pt}
\renewcommand{\footrulewidth}{0.4pt}
\setlength\headheight{57pt}
\lhead{Relazione del progetto di Ricerca Operativa}
\lfoot{Barzon Giacomo}
\rfoot{Pagina \thepage\ di \pageref{LastPage}}


%\newcommand{\code}[1]{\texttt{#1}}
\interfootnotelinepenalty=10000
\DeclareTextFontCommand{\code}{\ttfamily\hyphenchar\font=45\relax}

\begin{document}

	\begin{titlepage} % Suppresses displaying the page number on the title page
	\begin{center} % Centre everything on the page

		%------------------------------------------------
		%	Headings
		%------------------------------------------------
		
		\textsc{\LARGE Università degli studi di Padova}\\[1.5cm]
		
		\textsc{\Large Corso di Laurea in Informatica}\\[0.5cm]
		
		\textsc{\Large Progetto di Ricerca Operativa}\\[0.5cm]
		

		
		\textsc{\large a.a. 2018/2019}\\[0.5cm] % Minor heading such as course title
		
		%------------------------------------------------
		%	Title
		%------------------------------------------------
		
		{\huge\ Relazione}\\[0.4cm] % Title of your document
		
		%------------------------------------------------
		%	Author(s)
		%------------------------------------------------
		
	 	\large
			\begin{tabular}{l r}
				Giacomo \textsc{Barzon} & 1143164 \\
			\end{tabular}
	\end{center}
\end{titlepage}
	%----------------------------------------------------------------------------------------

	\newpage
	
	\tableofcontents
	
	\newpage
	
	\section{Problema}
	Un azienda siderurgica è specializzata nella produzione di tre componenti fondamentali per la costruzione di motori auto, che noi chiameremo come A,B e C per comodità.\\
	Essa attualmente dispone di:
	\begin{itemize}
		\item 200 unità "grezze" di A;
		\item 300 unità "grezze" di B;
		\item 100 unità "grezze" di C.
	\end{itemize}
	Esse tuttavia non sono ancora pronte per essere vendute. Ogni unità infatti, prima di poter essere considerata come completata, e quindi pronta ad essere commercializzata, deve attraversare 3 fasi sequenziali:
	\begin{itemize}
		\item rifinitura;
		\item verniciatura;
		\item lucidatura;
	\end{itemize}
	Tra una lavorazione e l'altra ogni unità deve effettuare un periodo di riposo, per questo motivo non è possibile effettuare due lavorazioni sulla stessa unità lo stesso giorno.\\
	Attualmente l'azienda possiede una macchina distinta per ogni tipologia di lavorazione.
	All'interno della seguente tabella è possibile vedere, per ogni tipologia di componente, il numero di unità che ciascuna macchina riesce a completare mediamente in un ora.
	\begin{table}[H]
	\setlength{\tabcolsep}{15pt} % Default value: 6pt
	\renewcommand{\arraystretch}{2} % Default value: 1
		\begin{center}
			\begin{tabular}{|c|c|c|c|}
				\hline
				& \textbf{Rifinitura} & \textbf{Verniciatura} & \textbf{Lucidatura} \\ \hline
				\textbf{A} &      10      &      8      &    11        \\ \hline
				\textbf{B} &       8     &       7     &      9      \\ \hline
				\textbf{C} &        13    &      10      &    15       \\ \hline
			\end{tabular}
		\end{center}
	\end{table}
	Ogni macchina non può lavorare al di fuori delle normali 8 ore lavorative in cui è aperta l'azienda ogni giorno.
	Ogni macchina viene utilizzata per effettuare lavorazioni su ogni tipologia di componente. Se durante il corso della giornata si rivelasse necessario passare dalla lavorazione di una tipologia di componente ad un altra è necessario effettuare un periodo di configurazione di un ora in cui la macchina interessata è inutilizzabile.
	Ogni macchina inoltre presenta un costo orario, in particolare:
	\begin{itemize}
		\item Il macchinario atto alla rifinitura costa 15 euro all'ora;
		\item Il macchinario atto alla verniciatura costa 25 euro all'ora;
		\item Il macchinario atto alla lucidatura costa 10 euro all'ora.
	\end{itemize}
	Attualmente l'azienda ha stipulato un contratto con un azienda la quale paga:
	\begin{itemize}
		\item 30 per ogni unità di A;
		\item 25 per ogni unità di B;
		\item 40 per ogni unità di C.		
	\end{itemize}
	Essa inoltre richiede che per l'inizio della prossima settimana le venga recapitata una quantità minima di:
	\begin{itemize}
		\item 20 unità di A;
		\item 40 unità di B;
		\item 10 unità di C.
	\end{itemize}
	Per ogni unità non recapitata l'azienda è costretta a pagare una penale di 35 euro.\\
	Definire un modello matematico che permetta di determinare il miglior processo produttivo ai fini di massimizzare i guadagni considerando che l'azienda dispone di un'intera settimana lavorativa (da lunedì a venerdì) per completare la lavorazione delle componenti "grezze" di cui dispone.
	\pagebreak
	\section{Modello Matematico}
	\subsection{Variabili decisionali}
	\begin{itemize}
		\item \( x_{p,s,g} : p \in \{A, B, C\},\ s \in \{grezzo, rifinito, verniciato, lucidato\},\ g \in \{lun, mar, mer, gio, ven\}\): Numero di ore in cui vengono lavorati pezzi di tipo p all'interno della macchina che si occupa di far passare l'unità dallo stato s al successivo il giorno g.
		\item \( y_{p,s,g} : p \in \{A, B, C\},\ s \in \{grezzo, rifinito, verniciato, lucidato\},\ g \in \{lun, mar, mer, gio, ven\}\): Numero di pezzi di tipo p che sono nello stato s durante il giorno g.
		\item\(ma_p : p \in \{ A,B,C\} \): Numero di pezzi di tipo p non prodotti entro la fine della settimana.
		\item \( z_{p,s,g} : p \in \{A, B, C\},m \in \{grezzo, rifinito, verniciato, lucidato\},g \in \{lun, mar, mer, gio, ven\}\): Variabile binaria che vale:
		\[ 
			z_{p,s,g}=
			\begin{sistema} 
				\textrm{1 \quad se la macchina che si occupa di far passare le unità dallo stato }\\
				\textrm{\ \ \quad al successivo ha lavorato a pezzi di tipo p il giorno g}\\ \\
				\textrm{0 \quad altrimenti} \\ 
			\end{sistema} 
		\]
		\item \( k_{s,g} : s \in \{grezzo, rifinito, verniciato, lucidato\}, g \in \{lun, mar, mer, gio, ven\} \) : Variabile binaria che vale:
		\[ 
			k_{s,g}=
			\begin{sistema} 
				\textrm{1 \quad se la macchina che si occupa di far passare le unità dallo stato}\\
				\textrm{\ \ \quad al successivo ha lavorato su almeno una tipologia di componente il giorno g}\\ \\
				\textrm{0 \quad altrimenti}
			\end{sistema} 
		\]
	\end{itemize}
	\subsection{Parametri}
	\begin{itemize}
		\item \( v_p : p \in \{A, B, C\} \): prezzo di vendita della componente p;
		\item \( c_s : s \in \{grezzo, rifinito, verniciato, lucidato\} \): costo orario della macchina s ;
		\item \( po_{p,s} : p \in \{A, B, C\},\ s \in \{grezzo, rifinito, verniciato, lucidato\}\): produzione oraria della componente p della macchina che si occupa di far passare l'unità dallo stato s al successivo;
		\item \( m_p : p \in \{A, B, C\}\): numero minimo di componenti p richieste;
		\item \(p_p\) : costo penale componente p;
		\item \(s_p : p \in \{A, B, C\} \): numero di unità grezze di partenza;
		\item \(o \) : Numero di ore massimo in cui è possibile utilizzare i macchinari.
	\end{itemize}
	\subsection{Funzione obbiettivo}
	$$
		max 
		\underbrace{\sum\limits_{\forall p} (v_p*y_{p,lucidato,ven})}_\textrm{Guadagni}
		-
		\underbrace{\sum\limits_{\forall m}(c_m \sum\limits_{\forall p,g}x_{pmg})}_\textrm{Costi}
		-
		\underbrace{\sum\limits_{\forall p}p_p*ma_p}_\textrm{Penale}
	$$
	\subsection{Vincoli}
	\begin{itemize}
		\item Attivazione variabile \( y_{p,s,g} \) con s e g di ordine maggiore ad 1:
		$$
			\forall p,s>1,g>1:y_{p,s,g}= 
			\underbrace{y_{p,s,g-1}}_{ \substack{\textrm{Numero unità }\\ \textrm{del} \\ \textrm{del giorno prima}}}
			+
			\underbrace{po_{p,s-1} \cdot x_{p,s-1,g}}_{\substack{\textrm{Numero unità}\\ \textrm{prodotte oggi}}}
			-
			\underbrace{po_{p,s} \cdot x_{p,s,g}}_{\substack{\textrm{Numero unità }\\ \textrm{stato completato} \\ \textrm{oggi}}}
		$$
		\item Attivazione variabile \( y_{p,s,g} \) con s e g di ordine uguale ad 1:

		$$
			\forall p : y_{p,\ grezzo,\ lun}=s_p-po_p,\ grezzo * x_p,\ grezzo,\ lun
		$$
		\item Attivazione variabile \( y_{p,s,g} \) con s di ordine uguale ad 1:
		$$
			\forall p,g>1 :
			y_{p,\ grezzo,\ g}=y_{p,\ grezzo,\ g-1}-po_{p,\ grezzo}*x_{p,\ grezzo,\ g}
		$$
		\item Attivazione variabile \( y_{p,s,g} \) con ordine di g uguale ad 1 ed s pari a 2:
		$$
		\forall p : y_{p,\ rifinito,\ lun}=po_{p,\ grezzo}* x_{p,\ grezzo,\ lun}
		$$
		\item Attivazione variabile \( y_{p,s,g} \) con ordine di g uguale ad 1 ed s maggiore a 2:
		$$
		\forall p,s>2 : y_{p,\ s, lun}=0;
		$$
		\item Numero di prodotti che hanno completato un determinato stato deve essere inferiore al numero di prodotti che avevo nello stato precedente il giorno prima:
		$$
			\forall p,s>1,g>1 : y_{p,s,g} \leq y_{p,s-1,g-1}
		$$
		\item Come regola precedente ma nel caso in cui si utilizza \(lun \in Giorni\) e \(rifinito \in Stati\):
		$$
		\forall p : y_{p,\ rifinito,\ lun}<=y_{p,\ grezzo,\ lun};
		$$
		\item Attivazione variabile \( z_{p,s,g} \):
		$$
			\forall p,s,g : x_{p,s,g}\leq z_{p,s,g}*M
		$$
		\item Attivazione variabile \( k_{p,s,g} \):
		$$
			\forall s,g : \sum\limits_{\forall p} x_{p,s,g}\leq k_{s,g}*M
		$$
		\item Attivazione variabile \(ma_p\):
		$$
			\forall p : y_{p,\ lucidato,\ ven}+ma_p \geq m_p
		$$
		\item Numero di ore lavorative di ogni macchina inferiore ad \emph{o} ore:
		$$
			\forall s,g : 
			\underbrace{\sum\limits_{\forall p}x_{p,s,g}}_{\substack{\textrm{Somma complessiva}\\ \textrm{ore macchina}}}
			+
			\underbrace{\sum\limits_{\forall p}z_{p,s,g}-k_{s,g}}_{\substack{\textrm{Numero di}\\ \textrm{cambi produzione} \\ \textrm{effettuati}}}
			\leq o
		$$
	\end{itemize}
\pagebreak
\section{Codice AMPL}
\subsection{Modello}
Viene qui proposto il file .mod utilizzato per rappresentare nel linguaggio AMPL il modello matematico descritto all'interno della sezione precedente.
\begin{lstlisting}
set Prodotti;
set Giorni ordered;
set Stati ordered;

#Dichiarazione Parametri

#Prezzi di vendita delle componenti
param v{Prodotti};
#Costo orario delle macchine
param c{Stati};
#produzione orari delle macchine in relazione a ciascun prodotto
param po{Prodotti, Stati};
#Numero minimo unita' richieste dal cliente
param minUnit{Prodotti};
#Penali da pagare per ogni unita' di prodotto non recapitata
param penale{Prodotti};
#Numero di unita' grezze di partenza
param startingUnits{Prodotti};
#Numero di ore massimo in cui e' possibile utilizzare i macchinari
param o;

#
param M=8;

#Dichiarazione Variabili

#Numero di ore in cui il macchinario e' attivo 
#su un determinato prodotto un determinato giorno
var x{Prodotti, Stati, Giorni}>=0;
#Numero di componenti di un determinato tipo in
#un determinato stato prodotti durante uno specifico giorno
var y{Prodotti, Stati, Giorni}>=0 integer;
#Numero di unita' non recapitate
var ma{Prodotti}>=0 integer;
#Variabile binaria che vale 1 se la macchina 
#ha lavorato ad un prodotto p il giorno g 0 altrimenti
var z{Prodotti, Stati, Giorni} binary;
#Variabile binaria che vale 1 se la macchina 
#ha lavorato ad un qualsiasi prodotto il giorno g 0 altrimenti
var k{Stati, Giorni} binary;

#Funzione obbiettivo
maximize profitto: sum{p in Prodotti} v[p]*y[p,last(Stati),last(Giorni)]
-sum{m in Stati, p in Prodotti, g in Giorni} c[m]*x[p,m,g]
-sum{p in Prodotti}penale[p]*ma[p]
;

#Attivazione della variabile Y  per ogni 
#elemento di Giorni e Stati con ordine maggiore di 1
s.t. attivazioneY
{p in Prodotti, g in Giorni, s in Stati: ord(s)>1 and ord(g)>1}:
y[p,s,g]=y[p,s,prev(g)]+po[p,prev(s)]*x[p,prev(s),g]
-po[p,s]*x[p,s,g];

#Attivazione della variabile Y con 
#elementi di Giorni e stati con ordine uguale ad 1
s.t. attivazioneYS0G0{p in Prodotti}:
y[p,first(Stati),first(Giorni)]=startingUnits[p]
-po[p,first(Stati)]*x[p,first(Stati),first(Giorni)];

#Attivazione della variabile y con 
#l'elemento di Stati con ordine uguale ad 1
s.t. attivazioneYS0{p in Prodotti, g in Giorni:ord(g)>1}:
y[p,first(Stati),g]=y[p,first(Stati),prev(g)]
-po[p,first(Stati)]*x[p,first(Stati),g];

#Attivazione della variabile y con l'elemento 
#di Giorni con ordine uguale ad 
#1 e quello di stati con ordine uguale a 2
s.t. attivazioneYGOS2{p in Prodotti}:
y[p,member(2,Stati),first(Giorni)]=
po[p,first(Stati)]*x[p,first(Stati),first(Giorni)];

#Attivazione della variabile y con l'elemento di Giorni 
#con ordine uguale ad 1 e quello di stati 
#con ordine uguale maggiore a 2
s.t. attivazioneYGO{p in Prodotti, s in Stati:ord(s)>2}:
y[p,s,first(Giorni)]=0;

#Il numero di prodotti lavorati non puo' superare il numero di 
#prodotti che si aveva il giorno prima nello stato precedente
s.t. maxLav{p in Prodotti, g in Giorni, s in Stati:ord(s)>1 and ord(g)>1}: 
y[p,s,g]<=y[p,prev(s),prev(g)]; 

#Definizione della regola precedente ma nel caso 
#in cui si utilizza l'elemento di giorni con cardinalita' 
#pari ad 1 e di Stati con cardinalita' pari a 2
s.t. maxLavYGOS2{p in Prodotti}:
y[p,member(2,Stati),first(Giorni)]<=y[p,first(Stati),first(Giorni)];

#Attivazione della variabile Ma
s.t. attivazioneMa{p in Prodotti}:
y[p,last(Stati), last(Giorni)]+ma[p]>=minUnit[p];

#Attivazione della variabileZ
s.t. attivazioneZ{p in Prodotti, g in Giorni, m in Stati}:
x[p,m,g] <=z[p,m,g]*M;

#Attivazione della variabile K
s.t. attivazioneK{g in Giorni, m in Stati}:
sum{p in Prodotti}x[p,m,g]<=k[m,g]*M;

#Numero massimo di ore in cui la macchina puo' essere attiva
s.t. maxOre{g in Giorni, m in Stati}:
sum{p in Prodotti} x[p,m,g]+sum{p in Prodotti}z[p,m,g]-k[m,g]<=o;

\end{lstlisting}
\subsection{Dati}
\subsubsection{Caso principale}
Viene qui proposto il file .dat relativo al problema principale discusso precedentemente.
\begin{lstlisting}
set Prodotti := A B C;
set Giorni := lun mar mer gio ven;
set Stati:= grezzo rifinito vernciato lucidato;

param     v :=
A		  30
B		  25
C		  40
;


param    c :=
grezzo   15
rifinito		 10
vernciato		 25
lucidato		 0
;

param 	 po :    grezzo    rifinito    vernciato    lucidato :=
A                10        8    	   11   		0         
B                8         7    	   9    		0
C                13        10   	   5   		    0
;

param    minUnit :=
A		 20
B		 40
C		 10
;

param 	 penale :=
A		 35
B		 35
C		 35
;

param 	 startingUnits :=
A		 200
B	  	 300
C		 150
;

param o :=8;
\end{lstlisting}
Il risultato dell'esecuzione del programma coi dati forniti da come risultato un profitto massimo di 2537.
Per ottenere il seguente risultato è fondamentale seguire il seguente processo produttivo.\\
All'interno della seguente tabella è descritto come suddividere durante la settimana la produzione relativa al prodotto A:
\begin{table}[H]
	\centering
	\setlength{\tabcolsep}{15pt} % Default value: 6pt
	\renewcommand{\arraystretch}{2} % Default value: 1
	\begin{center}
	\begin{tabular}{|l|l|l|l|l|}
		\hline
		& \textbf{Grezzo} & \textbf{Rifinito} & \textbf{Verniciato} & \textbf{Lucidato} \\ \hline
		\textbf{Lunedì}    & 8               & 0                 & 0                   & 0                 \\ \hline
		\textbf{Martedì}   & 1.2               & 5.75              & 0                   & 0                 \\ \hline
		\textbf{Mercoledì} & 2             & 2                 & 2             & 0                 \\ \hline
		\textbf{Giovedì}   & 0               & 1.25              & 0                   & 0                 \\ \hline
		\textbf{Venerdì}   & 0               & 0                 & 2.545               & 0                 \\ \hline
	\end{tabular}
	\end{center}
\end{table}

All'interno della seguente tabella è descritto come suddividere durante la settimana la produzione relativa al prodotto B:
\begin{table}[H]
	\centering
	\setlength{\tabcolsep}{15pt} % Default value: 6pt
	\renewcommand{\arraystretch}{2} % Default value: 1
	\begin{center}
		\begin{tabular}{|l|l|l|l|l|}
			\hline
			& \textbf{Grezzo} & \textbf{Rifinito} & \textbf{Verniciato} & \textbf{Lucidato} \\ \hline
			\textbf{Lunedì}    & 0              & 0                 & 0                   & 0                 \\ \hline
			\textbf{Martedì}   & 0               & 0              & 0                   & 0                 \\ \hline
			\textbf{Mercoledì} & 5             & 0                 & 0             & 0                 \\ \hline
			\textbf{Giovedì}   & 0               & 5.71429              & 0                   & 0                 \\ \hline
			\textbf{Venerdì}   & 0               & 0                 & 4.44444               & 0                 \\ \hline
		\end{tabular}
	\end{center}
\end{table}

All'interno della seguente tabella è descritto come suddividere durante la settimana la produzione relativa al prodotto C:
\begin{table}[H]
	\centering
	\setlength{\tabcolsep}{15pt} % Default value: 6pt
	\renewcommand{\arraystretch}{2} % Default value: 1
	\begin{center}
		\begin{tabular}{|l|l|l|l|l|}
			\hline
			& \textbf{Grezzo} & \textbf{Rifinito} & \textbf{Verniciato} & \textbf{Lucidato} \\ \hline
			\textbf{Lunedì}    & 0               & 0                 & 0                   & 0                 \\ \hline
			\textbf{Martedì}   & 5.76923               & 0              & 0                   & 0                 \\ \hline
			\textbf{Mercoledì} & 0             & 5                 & 0             & 0                 \\ \hline
			\textbf{Giovedì}   & 0               & 0              & 5                   & 0                 \\ \hline
			\textbf{Venerdì}   & 0               & 0                 & 0               & 0                 \\ \hline
		\end{tabular}
	\end{center}
\end{table}
\subsubsection{Casi alternativi}
All'interno di questa sezione verranno presentati alcuni file .dat relativi ad alcune istanze alternative del problema.
\paragraph{Penale bassa}\mbox{}\\
Analizziamo ora un caso alternativo in cui i costi relativi alla penale sono sufficientemente bassi da invogliare il solver a focalizzarsi solo ed esclusivamente sulla componente più redditizia piuttosto che sul cercare di produrre il numero minimo di componenti richieste dal cliente.\\
Le principali modifiche che sono state effettuate al problema originale sono le seguenti:
\begin{itemize}
	\item Sono stati abbassati i costi relativi alle penali
	\item Sono stati alzati guadagni e produzione oraria in modo tale da rendere più evidente i risultati.
	\item Sono stati alzati notevolmente il numero di componente grezze che si dispone.
\end{itemize}
\begin{lstlisting}
set Prodotti := A B C;
set Giorni := lun mar mer gio ven;
set Stati:= grezzo rifinito vernciato lucidato;

param     v :=
A		  150
B		  100
C		  300
;


param    c :=
grezzo   		 100
rifinito		 50
vernciato		 150
lucidato		 0
;

param 	 po :    grezzo    rifinito    vernciato    lucidato :=
A                50        70    	   130   		0         
B                70        90    	   160    		0
C                30        50   	   100   		0
;

param    minUnit :=
A		 50
B		 100
C		 30
;

param 	 penale :=
A		 70
B		 60
C		 130
;

param 	 startingUnits :=
A		 3000
B	  	 3000
C		 3000
;

param o :=8;
\end{lstlisting}
I risultati ottenuti sono molto interessanti, infatti aumentando anche solo di poco i costi delle penali, il solver deciderà di consegnare almeno il minimo di numero di unità minimo richiesto.\\
Dunque è possibile osservare come prima che possa divenire proficua la strategia descritta precedentemente è necessario che i costi relativi alle penali siano all'incirca pari alla meta' dei guadagni derivati dalla vendita.\\
Il risultato dell'esecuzione del programma coi dati forniti da come risultato un profitto massimo di 154700 euro. Inoltre sarà necessario pagare una penale pari a 9500 euro.
Per ottenere il seguente risultato è fondamentale seguire il seguente processo produttivo.\\
All'interno della seguente tabella è descritto come suddividere durante la settimana la produzione relativa al prodotto A:
\begin{table}[H]
	\centering
	\setlength{\tabcolsep}{15pt} % Default value: 6pt
	\renewcommand{\arraystretch}{2} % Default value: 1
	\begin{center}
		\begin{tabular}{|l|l|l|l|l|}
			\hline
			& \textbf{Grezzo} & \textbf{Rifinito} & \textbf{Verniciato} & \textbf{Lucidato} \\ \hline
			\textbf{Lunedì}    & 0               & 0                 & 0                   & 0                 \\ \hline
			\textbf{Martedì}   & 0               & 0              	 & 0                   & 0                 \\ \hline
			\textbf{Mercoledì} & 0             & 0                 & 0             & 0                 \\ \hline
			\textbf{Giovedì}   & 0               & 0              & 0                   & 0                 \\ \hline
			\textbf{Venerdì}   & 0               & 0                 & 0               & 0                 \\ \hline
		\end{tabular}
	\end{center}
\end{table}

All'interno della seguente tabella è descritto come suddividere durante la settimana la produzione relativa al prodotto B:
\begin{table}[H]
	\centering
	\setlength{\tabcolsep}{15pt} % Default value: 6pt
	\renewcommand{\arraystretch}{2} % Default value: 1
	\begin{center}
		\begin{tabular}{|l|l|l|l|l|}
			\hline
			& \textbf{Grezzo} & \textbf{Rifinito} & \textbf{Verniciato} & \textbf{Lucidato} \\ \hline
			\textbf{Lunedì}    & 0              & 0                 & 0                   & 0                 \\ \hline
			\textbf{Martedì}   & 0               & 0              & 0                   & 0                 \\ \hline
			\textbf{Mercoledì} & 0             & 0                 & 0             & 0                 \\ \hline
			\textbf{Giovedì}   & 0               & 0              & 0                   & 0                 \\ \hline
			\textbf{Venerdì}   & 0               & 0                 & 0               & 0                 \\ \hline
		\end{tabular}
	\end{center}
\end{table}

All'interno della seguente tabella è descritto come suddividere durante la settimana la produzione relativa al prodotto C:
\begin{table}[H]
	\centering
	\setlength{\tabcolsep}{15pt} % Default value: 6pt
	\renewcommand{\arraystretch}{2} % Default value: 1
	\begin{center}
		\begin{tabular}{|l|l|l|l|l|}
			\hline
			& \textbf{Grezzo} & \textbf{Rifinito} & \textbf{Verniciato} & \textbf{Lucidato} \\ \hline
			\textbf{Lunedì}    & 8               & 0                 & 0                   & 0                 \\ \hline
			\textbf{Martedì}   & 8               & 0              & 0                   & 0                 \\ \hline
			\textbf{Mercoledì} & 8             & 3.2                 & 0             & 0                 \\ \hline
			\textbf{Giovedì}   & 0               & 8              & 0                   & 0                 \\ \hline
			\textbf{Venerdì}   & 0               & 0                 & 5.6               & 0                 \\ \hline
		\end{tabular}
	\end{center}
\end{table}
\paragraph{Costi e risorse molto più stringenti}\mbox{}\\
Analizziamo ora un caso in cui costi relativi a macchine e penali sono stati resi molto più stringenti, inoltre sono state abbassate notevolmente il numero di componenti "grezze" a disposizione inizialmente.
\begin{lstlisting}
set Prodotti := A B C;
set Giorni := lun mar mer gio ven;
set Stati:= grezzo rifinito vernciato lucidato;

param     v :=
A		  30
B		  25
C		  40
;


param    c :=
grezzo   15
rifinito		 30
vernciato		 50
lucidato		 80
;

param 	 po :    grezzo    rifinito    vernciato    lucidato :=
A                10        8    	   11   		0         
B                8         7    	   9    		0
C                13        10   	   5   		    0
;

param    minUnit :=
A		 20
B		 40
C		 10
;

param 	 penale :=
A		 100
B		 100
C		 100
;

param 	 startingUnits :=
A		 40
B	  	 80
C		 20
;

param o :=8;
\end{lstlisting}
Il risultato dell'esecuzione del programma coi dati forniti da come risultato un profitto massimo di 1044 euro.
Per ottenere il seguente risultato è fondamentale seguire il seguente processo produttivo.\\
Le principali modifiche che sono state effettuate al problema originale sono le seguenti:
\begin{itemize}
	\item Sono stati innalzati i costi relativi alle macchine a circa il doppio del prezzo di vendita' di ciascuna componente;
	\item Sono stati innalzati i costi relativi alle penali;
	\item Sono state abbassate le componenti "grezze" a disposizione inizialmente a circa il doppio della quantita' minima richiesta dal cliente;
\end{itemize}
All'interno della seguente tabella è descritto come suddividere durante la settimana la produzione relativa al prodotto A:
\begin{table}[H]
	\centering
	\setlength{\tabcolsep}{15pt} % Default value: 6pt
	\renewcommand{\arraystretch}{2} % Default value: 1
	\begin{center}
		\begin{tabular}{|l|l|l|l|l|}
			\hline
			& \textbf{Grezzo} & \textbf{Rifinito} & \textbf{Verniciato} & \textbf{Lucidato} \\ \hline
			\textbf{Lunedì}    & 1.5               & 0                 & 0                   & 0                 \\ \hline
			\textbf{Martedì}   & 0.5               & 0              	 & 0                   & 0                 \\ \hline
			\textbf{Mercoledì} & 1.2             & 1.5                 & 0             & 0                 \\ \hline
			\textbf{Giovedì}   & 0               & 1.5              & 0.363636                   & 0                 \\ \hline
			\textbf{Venerdì}   & 0               & 0                 & 1.45455               & 0                 \\ \hline
		\end{tabular}
	\end{center}
\end{table}

All'interno della seguente tabella è descritto come suddividere durante la settimana la produzione relativa al prodotto B:
\begin{table}[H]
	\centering
	\setlength{\tabcolsep}{15pt} % Default value: 6pt
	\renewcommand{\arraystretch}{2} % Default value: 1
	\begin{center}
		\begin{tabular}{|l|l|l|l|l|}
			\hline
			& \textbf{Grezzo} & \textbf{Rifinito} & \textbf{Verniciato} & \textbf{Lucidato} \\ \hline
			\textbf{Lunedì}    & 0              & 0                 & 0                   & 0                 \\ \hline
			\textbf{Martedì}   & 5               & 0              & 0                   & 0                 \\ \hline
			\textbf{Mercoledì} & 3.25             & 3.71429                 & 0             & 0                 \\ \hline
			\textbf{Giovedì}   & 0               & 3.71429              & 1.33333                   & 0                 \\ \hline
			\textbf{Venerdì}   & 0               & 0                 & 3.11111               & 0                 \\ \hline
		\end{tabular}
	\end{center}
\end{table}

All'interno della seguente tabella è descritto come suddividere durante la settimana la produzione relativa al prodotto C:
\begin{table}[H]
	\centering
	\setlength{\tabcolsep}{15pt} % Default value: 6pt
	\renewcommand{\arraystretch}{2} % Default value: 1
	\begin{center}
		\begin{tabular}{|l|l|l|l|l|}
			\hline
			& \textbf{Grezzo} & \textbf{Rifinito} & \textbf{Verniciato} & \textbf{Lucidato} \\ \hline
			\textbf{Lunedì}    & 0               & 0                 & 0                   & 0                 \\ \hline
			\textbf{Martedì}   & 0.461538               & 0              & 0                   & 0                 \\ \hline
			\textbf{Mercoledì} & 0.769231             & 0.6                 & 0             & 0                 \\ \hline
			\textbf{Giovedì}   & 0               & 0.7              & 0.6                   & 0                 \\ \hline
			\textbf{Venerdì}   & 0               & 0                 & 1.4               & 0                 \\ \hline
		\end{tabular}
	\end{center}
\end{table}
E' interessante osservare come nel risultato ottenuto dal solver, si riesca a malapena ad arrivare al numero minimo di unità richieste dal cliente senza riuscire a produrre alcuna unità in eccesso.
\end{document}