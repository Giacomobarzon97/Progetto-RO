\documentclass[12pt]{article}
\usepackage[utf8]{inputenc}
\usepackage[italian]{babel}
\usepackage{graphicx}
\usepackage{geometry}
\usepackage{url}
\usepackage{amsmath, amsfonts, amssymb}
\usepackage[hang,flushmargin]{footmisc}
\usepackage{fancyhdr}
\usepackage{lastpage}
\usepackage{footnote}
\usepackage{float}
\usepackage{textcomp}
\usepackage[colorinlistoftodos]{todonotes}
%\usepackage[hidelinks]{hyperref}
\PassOptionsToPackage{hyphens}{url}\usepackage{hyperref}
\hypersetup{colorlinks, linkcolor=black, urlcolor=blue}
\newenvironment{sistema}% 
{\left\lbrace\begin{array}{@{}l@{}}}% 
	{\end{array}\right.}
%Layout di pagina
\geometry{
	a4paper,
	total={160mm,225mm},
	left=25mm,
	top=25mm
}

%intestazione e piè di pagina
\pagestyle{fancy}
\fancyhf{}
\renewcommand{\headrulewidth}{0.4pt}
\renewcommand{\footrulewidth}{0.4pt}
\setlength\headheight{57pt}
\lhead{Relazione del progetto di Ricerca Operativa}
\lfoot{Barzon Giacomo}
\rfoot{Pagina \thepage\ di \pageref{LastPage}}


%\newcommand{\code}[1]{\texttt{#1}}
\interfootnotelinepenalty=10000
\DeclareTextFontCommand{\code}{\ttfamily\hyphenchar\font=45\relax}

\begin{document}

	\begin{titlepage} % Suppresses displaying the page number on the title page
	\begin{center} % Centre everything on the page

		%------------------------------------------------
		%	Headings
		%------------------------------------------------
		
		\textsc{\LARGE Università degli studi di Padova}\\[1.5cm]
		
		\textsc{\Large Corso di Laurea in Informatica}\\[0.5cm]
		
		\textsc{\Large Progetto di Ricerca Operativa}\\[0.5cm]
		

		
		\textsc{\large a.a. 2018/2019}\\[0.5cm] % Minor heading such as course title
		
		%------------------------------------------------
		%	Title
		%------------------------------------------------
		
		{\huge\ Relazione}\\[0.4cm] % Title of your document
		
		%------------------------------------------------
		%	Author(s)
		%------------------------------------------------
		
	 	\large
			\begin{tabular}{l r}
				Giacomo \textsc{Barzon} & 1143164 \\
			\end{tabular}
	\end{center}
\end{titlepage}
	%----------------------------------------------------------------------------------------

	\newpage
	
	\tableofcontents
	
	\newpage
	
	\section{Problema}
	Un azienda siderurgica produce tre delle principali componenti fondamentali per la realizzazione di motori per auto, le quali verranno chiamate come componenti A, B e C per comodità.\\
	Ogni componente per essere considerata come completata, e quindi pronta ad essere commercializzata, deve seguire uno specifico iter composto da 3 lavorazioni le quali devono essere eseguite obbligatoriamente in una specifica sequenza.\\
	Tra una lavorazione e l'altra ogni componente deve effettuare un periodo di riposo, per questo motivo non è possibile effettuare effettuare due lavorazioni sulla stessa unità lo stesso giorno.\\
	All'interno della seguente tabella è possibile vedere, per ogni tipologia di componente e lavorazione, il numero di unità che è possibile completare in un ora.
	
	\begin{table}[H]
		\setlength{\tabcolsep}{15pt} % Default value: 6pt
		\renewcommand{\arraystretch}{2} % Default value: 1
		\begin{center}
			\begin{tabular}{|c|c|c|c|}
				\hline
				& \textbf{Lavorazione 1} & \textbf{Lavorazione 2} & \textbf{Lavorazione 3} \\ \hline
				\textbf{A} &      10      &      8      &    11        \\ \hline
				\textbf{B} &       8     &       7     &      9      \\ \hline
				\textbf{C} &        13    &      10      &    15       \\ \hline
			\end{tabular}
		\end{center}
	\end{table}

	Attualmente l'azienda possiede solamente una macchina dedicata a ciascuna tipologia di lavorazione lavorazione. Ogni macchina può lavorare solo ed esclusivamente su una tipologia di componente per volta e per un massimo di 8 ore complessive a giorno.\\
	Ogni qualvolta sia necessario passare dalla lavorazione di un componente all'altro durante la giornata è necessario effettuare una configurazione del macchinario di circa un ora.\\
	Ogni macchina ha un costo orario, in particolare:
	\begin{itemize}
		\item il macchinario 1 costa 4 all'ora;
		\item il macchinario 2 costa 5 all'ora;
		\item il macchinario 3 costa 3 all'ora.
	\end{itemize}
	Attualmente l'azienda ha stipulato un contratto con un azienda la quale paga:
	\begin{itemize}
		\item 30 per ogni unità di componente 1
		\item 25 per ogni unità di componente 2
		\item 40 per ogni unità di componetne 3
	\end{itemize}
	Essa inoltre richiede settimanalmente una quantità minima di:
	\begin{itemize}
		\item 10 unità di componente 1;
		\item 15 unità di componente 2;
		\item 13 unità di componente 3.
	\end{itemize}
	Per ogni unità non recapitata l'azienda è costretta a pagare una penale di 35 euro.\\
	Definire un modello matematico che permetta di determinare il miglior processo produttivo ai fini di massimizzare i guadagni settimanali.
	\pagebreak
	\section{Modello Matematico}
	\subsection{Variabili decisionali}
	\begin{itemize}
		\item \( x_{pmg} : p \in \{A, B, C\},\ m \in \{1, 2, 3\},\ g \in \{1, 2, 3, 4, 5, 6, 7\}\): Numero di ore in cui vengono lavorati pezzi di tipo p all'interno della macchina m nel giorno g.
		\item \( y_{psg} : p \in \{A, B, C\},\ s \in \{0, 1, 2, 3\},\ g \in \{1, 2, 3, 4, 5, 6, 7\}\): Numero di pezzi di tipo p che hanno terminato la lavorazione di tipo s durante il giorno g.
		\item\(ma_p : p \in \{ A,B,C\} \): Numero di pezzi di tipo p non prodotti entro la fine della settimana.
		\item \( z_{pmg} : p \in \{A, B, C\},\ m \in \{1, 2, 3\},\ g \in \{1, 2, 3, 4, 5, 6, 7\}\): Variabile binaria che vale:
		\[ 
			z_{pmg}=
			\begin{sistema} 
				\textrm{1 \quad se la macchina m ha lavorato a pezzi di tipo p il giorno g}\\
				\textrm{0 \quad altrimenti} \\ 
			\end{sistema} 
		\]
		\item \( k_{mg} : m \in \{1,2,3\}, g \in \{1,2,3,4,5,6,7\} \) : Variabile binaria che vale:
		\[ 
			k_{pmg}=
			\begin{sistema} 
				\textrm{1 \quad se la macchina m  ha lavorato su almeno una tipologia di componente il giorno g}\\
				\textrm{0 \quad altrimenti}
			\end{sistema} 
		\]
	\end{itemize}
	\subsection{Parametri}
	\begin{itemize}
		\item \( v_p : p \in \{A, B, C\} \): prezzo di vendita della componente p.
		\item \( c_m : m \in \{1, 2, 3\} \): costo orario della macchina m
		\item \( po_{pm} : p \in \{A, B, C\},\ m \in \{1, 2, 3\}\): produzione oraria della componente p all'interno della macchina m.
		\item \( m_p : p \in \{A, B, C\}\): numero minimo di componenti p richieste.
		\item \(p_p\): costo penale componente p.
	\end{itemize}
	\subsection{Funzione obbiettivo}
	$$
		max 
		\underbrace{\sum\limits_{\forall p} (v_p \sum\limits_{\forall g} y_{ps^*g})}_\textrm{Guadagni}
		-
		\underbrace{\sum\limits_{\forall m}(c_m \sum\limits_{\forall p,g}x_{pmg})}_\textrm{Costi}
		-
		\underbrace{\sum\limits_{\forall p}p_p*ma_p}_\textrm{Penale}
	$$
	Dove \(s^*\) rappresenta l'ultima lavorazione che ogni componente deve superare per essere completata. 
	\subsection{Vincoli}
	\begin{itemize}
		\item Attivazione variabile \( y_{p,s,g} \):
		$$
			\forall p,s,g:y_{p,s,g}= 
			\underbrace{y_{p,s,g-1}}_{ \substack{\textrm{Numero unità }\\ \textrm{del} \\ \textrm{del giorno prima}}}
			+
			\underbrace{po_{p,s-1} \cdot x_{p,s-1,g}}_{\substack{\textrm{Numero unità}\\ \textrm{prodotte oggi}}}
			-
			\underbrace{po_{p,s} \cdot x_{p,s,g}}_{\substack{\textrm{Numero unità }\\ \textrm{stato completato} \\ \textrm{oggi}}}
		$$
		\item Numero di prodotti che hanno completato un determinato stato deve essere inferiore al numero di prodotti che avevo nello stato precedente il giorno prima:
		$$
			\forall p,s>1,g>1 : y_{p,s,g} \leq y_{p,s-1,g-1}
		$$
		\item Attivazione variabile \( z_{p,m,g} \):
		$$
			\forall p,m,g : x_{p,m,g}\leq z_{p,m,g}*M
		$$
		\item Attivazione variabile \( k_{p,m,g} \):
		$$
			\forall m,g : \sum\limits_{\forall p} x_{p,m,g}\leq z_{m,g}*M
		$$
		\item Attivazione variabile \(ma_p\):
		$$
			\forall p : y_{p,s^*,g^*}+ma_p \geq m_p
		$$
		Dove \(s^*\) indica l'ultimo stato in cui è possibile che risieda la componente p e \(g*\) indica l'ultimo giorno della settimana.
		\item Numero di ore lavorative di ogni macchina inferiore ad 8 ore:
		$$
			\forall m,g : 
			\underbrace{\sum\limits_{\forall p}x_{p,m,g}}_{\substack{\textrm{Somma complessiva}\\ \textrm{ore macchina}}}
			+
			\underbrace{\sum\limits_{\forall p}z_{p,m,g}-k_{m,g}}_{\substack{\textrm{Numero di}\\ \textrm{cambi produzione} \\ \textrm{effettuati}}}
			\leq 8
		$$
	\end{itemize}
\end{document}
