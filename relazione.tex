\documentclass[12pt]{article}
\usepackage[utf8]{inputenc}
\usepackage[italian]{babel}
\usepackage{graphicx}
\usepackage{geometry}
\usepackage{url}
\usepackage[hang,flushmargin]{footmisc}
\usepackage{fancyhdr}
\usepackage{lastpage}
\usepackage{footnote}
\usepackage{float}
\usepackage{textcomp}
\usepackage[colorinlistoftodos]{todonotes}
%\usepackage[hidelinks]{hyperref}
\PassOptionsToPackage{hyphens}{url}\usepackage{hyperref}
\hypersetup{colorlinks, linkcolor=black, urlcolor=blue}

%Layout di pagina
\geometry{
	a4paper,
	total={160mm,225mm},
	left=25mm,
	top=25mm
}

%intestazione e piè di pagina
\pagestyle{fancy}
\fancyhf{}
\renewcommand{\headrulewidth}{0.4pt}
\renewcommand{\footrulewidth}{0.4pt}
\setlength\headheight{57pt}
\lhead{Relazione del progetto \textbf{DevSpace}}
\lfoot{DevSpace}
\rfoot{Pagina \thepage\ di \pageref{LastPage}}


%\newcommand{\code}[1]{\texttt{#1}}
\interfootnotelinepenalty=10000
\DeclareTextFontCommand{\code}{\ttfamily\hyphenchar\font=45\relax}

\begin{document}

	\begin{titlepage} % Suppresses displaying the page number on the title page
	\begin{center} % Centre everything on the page

		%------------------------------------------------
		%	Headings
		%------------------------------------------------
		
		\textsc{\LARGE Università degli studi di Padova}\\[1.5cm]
		
		\textsc{\Large Corso di Laurea in Informatica}\\[0.5cm]
		
		\textsc{\Large Progetto di Ricerca Operativa}\\[0.5cm]
		

		
		\textsc{\large a.a. 2018/2019}\\[0.5cm] % Minor heading such as course title
		
		%------------------------------------------------
		%	Title
		%------------------------------------------------
		
		{\huge\ Relazione}\\[0.4cm] % Title of your document
		
		%------------------------------------------------
		%	Author(s)
		%------------------------------------------------
		
	 	\large
			\begin{tabular}{l r}
				Giacomo \textsc{Barzon} & 1143164 \\
			\end{tabular}
	\end{center}
\end{titlepage}
	%----------------------------------------------------------------------------------------

	\newpage
	
	\tableofcontents
	
	\newpage
	
	\section{Problema}
	Un azienda siderurgica possiede 4 fabbriche dislocate in punti diversi della città. Ogni fabbrica produce principalmente 3 prodoti distinti in quantità diverse.
	Negli ultimi mesi è stata effettuata una ricerca molto approffondita sulla produttività di ciascuna fabbrica permettendo così di realizzare una stima abbastanza precisa del numero di unità prodotte in ciascuna fabbrica ogni mese. \\
	La stima è riassunta in modo tabellare di seguito.\\
	TABELLA\\
	L'azienda di recente inoltre è entrata in contatto con quattro  nuovi clienti i quali per ogni unità di prodotto a loro recapitata sono disposti a pagare le seguenti quantità di denaro.\\
	TABELLA\\
	Tuttavia ogni azienda per poter stabilire un contratto richiede che le venga fornita una quantità minima di unità di prodotti. Più nello specifico:
	\begin{itemize}
		\item L'azienda 1 richiede X unità di A,  X unità di B ed X unità di C.
		\item L'azienda 2 richiede X unità di A,  X unità di B ed X unità di C
		\item L'azienda 3 richiede X unità di A,  X unità di B ed X unità di C
		\item L'azienda 4 richiede X unità di A,  X unità di B ed X unità di C
	\end{itemize}
	Inoltre l'Azienda 1 e 3 sono rivali ed hanno stabilito che nessuna delle due intende firmare un contratto se viene rifornita anche l'azienda rivale.\\
	Di seguito sono riportati i costi necessari per spedire un camion pieno di prodotti dalle sede delle aziende alla sede più vicina dei clienti che si vuole rifornire.\\
	TABELLA\\
	Considerando che un camion utilizzato per effettuare le spedizioni può contenere al massimo X unità di A e che B occupa la metà dello spazio di A mentre al contrario C ne occupa il doppio, si richiede di massimizzare il guadagno complessivo dell'azienda.

\end{document}
